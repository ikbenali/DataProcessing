\section{Problem 1: Design critique}

An short critique will be written for the Confluence visualisation by Harshawardhan Nene and Kedar Vaidya\footnote{\url{http://iibh.apphb.com/}}\\

The visualisations will be critiqued against the following points also shown in appendix 1

\begin{enumerate}[noitemsep]
    \item What is the problem domain or context of the visualisation?
    \item Which tasks can be achieved with this visualisation
    \item Consider Tufte\'s principles of graphical integrity
    \item Which of Tufte\'s visualisation design principles are adhered
    \item Describe how graphic design principles are used
    \item Comment on the visual encodings that are used
    \item Comment on subjective dimensions
    \item What is the intended goal of the visualization and is it achieved?
    \item Are there any things you would do differently, and why?
\end{enumerate}

\paragraph{1}\mbox{}\\
This visualisation visualises the differences of opinion between the audience and critics and attempts to map the financial succes of a movie to its ratings.

\paragraph{2}\mbox{}\\
The following tasks can be achieved:

\begin{itemize}[noitemsep]
    \item Find a correlation between high approval ratings from critics and the audience to the financial success.
    \item Relate if the critics are being over critical regarding different rating given by the audience.
    \item Find out if there is a relation between high financial budeting and high rating.
\end{itemize}

\paragraph{3}\mbox{}\\
The scale of the visualisation is unlabeled and there is no given legend to understand what the colors mean.\\ Furthermore there is no horizontal dimensions regarding in which year the movies are released.\\
There is not a significant lie factor due to the fact that there is sufficient separation between the user and critic rating and illustrates if the rating overlap or is significantly separated.

\paragraph{4}\mbox{}\\
When the option for fan art is disabled there still exist a high data ink ratio, meaning that there is a lot of data shown but for that data needs a lot of ink. Furthermore only one movie can be chosen. There is barely any chart junk and there exists a high data density.

\paragraph{5}\mbox{}\\
In this visualisation the contrast is used to show the differentiation between ratings. Due to the fact the same color is used for every movie there may exist a high repetition.\\
The data is closely shown and there also exists overlapping for some movies.

\paragraph{6}\mbox{}\\
The visual codings which is used primarily is color value and size. The color value is used to show the different ratings given by critics and the audience. The size is used to visualize the financial success. 

\paragraph{7}\mbox{}\\
Regarding the aesthetics,style,playfullness and vividness I would say that this visualisation is pretty dull and does not show a lot of variation.

\paragraph{8}\mbox{}\\
The intended goal was probably to provide the reader information about the discrepancy between critic rating and audience rating. In my opinion it is achieved but it is not clear immediatly.

\paragraph{9}\mbox{}\\
I would change the way the movie is represented and show it more along the lines of financial success, with an option to filter it through different genres. When the movie is clicked upon it may show the difference between audience and critics rating. Furthermore I would set the movies in chronological order of release.

